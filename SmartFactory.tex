\documentclass{article}
\usepackage[german]{babel}
\usepackage[utf8]{inputenc}
% Set title
\def\Title{Smart Factory}
\title{\Title}
\def\Author{Kevin Stahel und Markus Gachnang}
\author{\Author}
\def\Subject{Sind ''Smart Factories'' bereits Realität \\ oder nur ein Konzept auf Papier?}
\date{\today{}}

\PassOptionsToPackage{hyphens}{url}\usepackage{hyperref}
\def\UrlBreaks{\do\/\do-}
\hypersetup{
	linktoc=all,
    pdftitle={\Title},
    pdfsubject={\Subject},
    pdfauthor={\Author}
}
\usepackage{lastpage}
\usepackage{graphicx}
\usepackage{xcolor}
\usepackage{booktabs}
\usepackage{lipsum}
% geometry
\usepackage{geometry}
\geometry{a4paper,portrait,left={3cm},right={3cm},top={2cm},bottom={2cm},includehead,head={0.11\textwidth}}
% Colors
\definecolor{LightGray}{rgb}{.96,.92,.90}
\definecolor{Gray}{rgb}{0.79, 0.75, 0.73}
\definecolor{DarkGray}{rgb}{.46,.42,.40}
\definecolor{DarkestGray}{rgb}{.31,.28,.26}
% fancyhdr
\usepackage{fancyhdr}
\pagestyle{fancyplain}
\fancyhf{}
% set header/footer
\lhead{\includegraphics[width=0.2\textwidth]{ost}}
\chead{\textbf{\Title}\\\Author}
\rhead{\small\thepage\hspace{2pt}/\hspace{2pt}\pageref{LastPage}}
% Bibliographie
\usepackage{cite}
\def\BibTeX{{\rm B\kern-.05em{\sc i\kern-.025em b}\kern-.08em
    T\kern-.1667em\lower.7ex\hbox{E}\kern-.125emX}}
% This document uses 'limap'. Inforamtion are available at https://www.ctan.org/pkg/limap
\usepackage[german]{limap}
\usepackage{color}
% The macro \MapBlockLabelFont determines the font changing command to be used for typesetting the block label. The default is empty.
\renewcommand\MapBlockLabelFont{\color{DarkestGray}}
% The macro \MapParskip determines the vertical distance of the text from the separating rules. The default is 2ex.
\renewcommand\MapParskip{1ex}
% The macro \MapTitleFraction determines the part of the page width devoted to the block label area. It is a fraction in the range from 0 to 1. The default value of \MapTitleFraction is 0.2.
\renewcommand\MapTitleFraction{.2}
% This macro determines the part of the page width devoted to the text area. It is a fraction in the range from 0 to 1. The default value of \MapTextFraction is 0.75.
% \MapTitleFraction and \MapTextFraction should add up to something less or equal to 1. Otherwise you will get some “overfull hbox” messages.
\renewcommand\MapTextFraction{.8}
% The macro \MapRuleWidth determines the width of the rules drawn between blocks. It is defined as a macro containig a length. The default is 1pt.
% It can even be set to 0pt to suppress the lines at all
\renewcommand\MapRuleWidth{1pt}
% The macro \MapRuleStart can for instance be used to achieve colored rules. For this purpose we can include the package xcolor in the preamble and select a named color for the rule. 
\renewcommand\MapRuleStart{\color{Gray}}
% The macro \MapTitleFont determines the font changing command to be used when typesetting the title of a map. The default is \Large.
\renewcommand\MapTitleFont{\Large\bfseries}
\renewcommand\MapFont{\normalsize}
% This macro determines the font changing command to be used for typesetting the additional text after titles on followup pages of multi-page maps. The default value is \small.
\renewcommand\MapTitleContinuedFont{\normalsize}
\renewcommand\MapContinued{Fortsetzung}
% Every block in TOC
%\def\MapBlockStartHook{}
%\set\MapBlockStartHook\MapBlockTOC
\usepackage{titlesec}
\begin{document}
% print title
\begin{titlepage}
  \begin{center}
    \Huge\textbf{\Title}
    \par\bigskip\bigskip
    \large\Subject
    \normalsize
    \par\bigskip\bigskip
  \begin{tabular}{p{3cm} p{6cm}}
    \textcolor{DarkestGray}{Autoren} & \Author \\
  \end{tabular}
  \end{center}
\end{titlepage}
% print index
\begingroup
% Section
\titleformat
{\section} % command
[display] % shape
{\MapTitleFont} % format
{\thesection.} % label
{\MapParskip} % sep
{} % before-code
[\MapTitleContinuedFont] % after-code
  \setcounter{page}{1}
  \tableofcontents{}
\endgroup
\newpage
\begin{Map}{Abstract}
\Block{Definition}
Der Begriff ''Smart Factory'' kommt von der Hightech-Strategie der deutschen Bundesregierung als Teil des Zukunftsprojekts ''Industrie 4.0''.
\Block{Fragestellung}
Den Begriff von ''Smart Factory'' gibt es schon seit 2014, aber gibt es auch Fertigungsanlagen, die der Definition gerecht werden?
Wir versuchen, eine genaue Definition zu erarbeiten, welche unterscheidet, ob es sich bei der Fertigungsanlage auch um eine ''Smart Factory'' handelt. 
\end{Map}
\begin{Map}{Quellenverweis}
\Block{Kommentierte Quellenliste}
Im Zuge der ersten Recherchen wurde eine kommentierte Quellenliste erstellt, um als Ausgangspunkt der weiteren Untersuchungen des Themas zu fungieren. Diese Liste verschafft einem einen Überblick über das Thema von ''Smart Factory''. \\\medskip
\WideBlock{
\begin{tabular}{p{0.2cm}p{4.2cm}p{2cm}p{3.2cm}p{3.2cm}}\toprule
   & Quelle & Art & Inhalt & Eignung \\\midrule

 1 & 
 \url{https://www.plattform-i40.de/PI40/Navigation/DE/Industrie40/WasIndustrie40/was-ist-industrie-40.html} &
 Internet-Dokument / Filmdokument &
 Definition von ''Industrie 4.0'' vom Bundesministerium für Bildung und Forschung von Deutschland &
 Gibt einen Überblick von ''Industrie 4.0'' und den Zusammenhang zu ''Smart Factory''\\\midrule

 2 &
 Huhmann A., ''Industrie 4.0: Auf dem Weg zur wirklich smarten Factory'' entwickler.de - S\&S Media Support GmbH, 2019. \url{https://entwickler.de/online/iot/industrie-4-0-auf-dem-weg-zur-smart-factory-579911790.html} &
 Internet-Dokument / Zeitschriften-artikel &
 Stand der Dinge und Ausblick auf Industrie 4.0 &
 Beschreibt wie die ''Smart Factory'' Wirklichkeit werden und wie der aktuelle Stand der Dinge ist\\\midrule

 3& 
 Limited, Wipro, “In 5 Schritten zum Smart Manufacturing | MoreThanDigital,” MoreThanDigital, 2020. \url{https://morethandigital.info/in-5-schritten-zum-smart-manufacturing/} &
 Internet-Dokument / Zeitschriften-artikel & 
 Beschreibt das Vorgehen in 5 Schritten um zur ''Smart Manufacturing'' zu gelangen & 
 Gibt Definitionen zu ''Smart Factory'' und eingesetzte Hilfsmittel um eine solche aufzubauen \\\midrule
 
 4& 
 RICHARDS, G. and Grinsted, S., The logistics and supply chain toolkit: Over 100 tools for tansport, warehousing and inventory manageme, Third edition. ISBN: 9781789660852 & 
 Buch (Monographie) & 
 Beschreibt wie Logistik und Inventar gemanagt werden kann & 
 Gibt unter anderem auch Auskunft, wie dies vollautomatisch (wie in einer ''Smart Factory'') realisiert werden kann\\\midrule
\end{tabular}}

\WideBlock{
\begin{tabular}{p{0.2cm}p{4.2cm}p{2cm}p{3.2cm}p{3.2cm}}\toprule
   & Quelle & Art & Inhalt & Eignung \\\midrule
   
 5& 
 B. Meussen, „Anwendung von Industrie 4.0 in Forschung und Praxis“, Nordakademie - Hochschule der Wirtschaft, Elmshorn, Arbeitspapiere der Nordakademie 2015–03,  2015. [Online]. Verfügbar unter: \url{http://hdl.handle.net/10419/121298} & 
 Paper & 
 I & E\\\midrule
 6 & 
 
 T. Ionescu und M. Merz, ''Cyber-physische Produktion: Modelle und Inszenierung der Smart Factory'', AIS-Studien, 2018, doi: 10.21241/SSOAR.64876 &
 Paper & I & E\\\midrule
 
 7& 
 T. Schulz und Vogel Business Media GmbH \& Co. KG, Industrie 4.0 Potenziale erkennen und umsetzen. 2017. &
 Buch & 
 I & E\\\midrule
 
 8& 
 H. S. Kang u. a., ''Smart manufacturing: Past research, present findings, and future directions'', Int. J. of Precis. Eng. and Manuf.-Green Tech., Bd. 3, Nr. 1, S. 111–128, Jan. 2016, doi: 10.1007/s40684-016-0015-5 & 
 Paper & 
 I & E\\\midrule
 
 9& Q & A & I & E\\\midrule
 
10& Q & A & I & E\\\bottomrule

\end{tabular}}

\Block{Verwendete Verweise}
\begingroup
% Remove Titles eg "Literatur"
\renewcommand{\section}[2]{}%
% Reminder: Recreate "SmartFactory.bll" when "citavi/citavi.bib" gets changed => run BibTeX ([F11] in TexMaker)
\bibliography{citavi/citavi}
% Use the "IEEE standard" as style => "IEEEtran.bst"
\bibliographystyle{IEEEtran}
\endgroup
\end{Map}
\end{document}
