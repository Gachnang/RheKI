\documentclass[11pt,titlepage]{article}
\usepackage{ucs}
\usepackage[utf8x]{inputenc}
\usepackage[T1]{fontenc}
\usepackage[ngerman]{babel}
\usepackage{graphicx}
\usepackage{titlesec}
\usepackage{url}
\usepackage{lastpage}
\usepackage{listings}
\usepackage{color}
\usepackage{fancyhdr}
\usepackage{geometry}
\usepackage{wrapfig}
\usepackage{float}
\usepackage{subcaption}
\usepackage{hyperref}
\usepackage{ragged2e}
\usepackage{framed}
\usepackage{quoting}
\usepackage{lscape}
\usepackage[table]{xcolor}
\usepackage{pdfpages}
% Bibliographie
\usepackage{cite}
\def\BibTeX{{\rm B\kern-.05em{\sc i\kern-.025em b}\kern-.08em
    T\kern-.1667em\lower.7ex\hbox{E}\kern-.125emX}}
% remove current style and use fancyplain
\pagestyle{fancyplain}
\fancyhf{}
% remove rule/lines as well
\renewcommand{\headrulewidth}{0pt}
\renewcommand{\footrulewidth}{0pt}
% set papersize, magin and footersize
\geometry{a4paper,portrait,left={3cm},right={3cm},top={2cm},bottom={1cm},includefoot,foot={1cm}}
% set footer
\rfoot{Seite \thepage \hspace{1pt} von \pageref{LastPage}}
% define some colors
\definecolor{lightgray}{rgb}{.95,.95,.95}
\definecolor{shadecolor}{rgb}{.95,.95,.95}
\definecolor{darkgray}{rgb}{.4,.4,.4}
\definecolor{purple}{rgb}{0.65, 0.12, 0.82}
% set color and font of ''\url''
\renewcommand\UrlFont{\color{blue}\rmfamily\itshape}
% colorbox which can wrap lines
\newcommand\code[1]{\codehelp#1 \relax\relax}
\def\codehelp#1 #2\relax{\allowbreak\grayspace\codecolor{#1}\ifx\relax#2\else
 \codehelp#2\relax\fi}
\newcommand\codecolor[1]{\colorbox{lightgray}{\textcolor{black}{%
  \ttfamily\mystrut\smash{\detokenize{#1}}}}}
\def\mystrut{\rule[\dimexpr-\dp\strutbox+\fboxsep]{0pt}{%
 \dimexpr\normalbaselineskip-2\fboxsep}}
\def\grayspace{\hspace{0pt minus \fboxsep}}
% add ''\code'' to highligth single code lines
%\newcommand{\code}[1]{\wrapcolorbox[lightgray]{\ttfamily{#1}}}

% add ''\shadedquotation'' to highligth quoates
\newenvironment{shadedquotation}
 {\begin{shaded*}
  \quoting[leftmargin=0pt, vskip=0pt]
 }
 {\endquoting
 \end{shaded*}
}

% set title
\title{Smart Factory}
\author{Kevin Stahel und Markus Gachnang}
\date{\today{}}
% set parindent to 0px to remove it (Einrücken von neuer Absatz)
\setlength\parindent{0pt}
% ---------------------------------------------------------------------------
% begin Document
\begin{document}
% set font
\sffamily
% print title
\maketitle
\newpage
% print index
\tableofcontents{}
\setcounter{page}{1}
\newpage
% linksbündig
\RaggedRight
% kein brechen von Wörtern
\tolerance=1
\emergencystretch=\maxdimen
\hyphenpenalty=10000
\hbadness=10000
\section{Brainstorm}
\label{sec:Brainstorm}
  \url{https://refa.de/service/refa-lexikon/smart-factory}
  \begin{itemize}
    \item bezeichnet eine Produktionsumgebung, die sich selbst organisiert (Mensch muss in den Produktionsprozess nicht mehr eingreifen)
    \item Basis solcher intelligenten Fabriken sind sogenannte Cyber Physical Systems (CPS)
    \item CPS sind Systeme, bei denen informations- und softwaretechnische Komponenten mit mechanischen bzw. elektronischen Komponenten verbunden werden. 
  \end{itemize}


  \url{https://www.bimos.com/B/ch-de/news/3014/smart-factory---die-zukunftsvision-der-industrie-40}
  \begin{itemize}
    \item Die intelligente Fabrik, die sich selbst organisiert
    \item schlanke und optimierte Prozesse
    \item kürzere Produktionszeiten
    \item Produktion von Individualprodukten zu Preisen von Massenprodukten
    \item Steigerung der Produktivität
    \item automatisierte, effiziente Bestellprozesse
    \item geringerer Personalaufwand in der Produktion
    \item höhere Flexibilität in der Produktion
    \item kürzere Markteinführungszeiten für neue Produkte
    \item schnelle Anpassung an neue oder veränderte Produktanforderungen
    \item agile Reaktion des Produktionsprozesses auf Schwankungen im Marktbedarf
  \end{itemize}

  \url{https://www.sciencedirect.com/science/article/pii/S2351978918306577}
  \begin{itemize}
    \item South Korean and Swedish examples on manufacturing settings
    \item Since the field of Smart Manufacturing is still in a rather immature phase, there are no fully established
    definitions. However, one example of definition of Smart Manufacturing could be: ''a (fully)-integrated and
    collaborative manufacturing system that responds in real time to meet the changing demands and conditions in the
    factory, supply network, and customer needs.''
  \end{itemize}


  \subsection{Examples}
  \label{sec:Brainstorm_Examples}

  Spiele, bei denen man ''Smart Factories'' macht:
  \begin{itemize}
    \item Satisfactory 
    \item Factorio
    \item HaliGalli2
  \end{itemize}


Platinen löten / bestücken

Wie erwähnt in \cite{RICHARDS.2020}

\pagebreak

\section{Kommentierte Quellenliste}
\label{sec:KommentierteQuellenliste}
  
\begingroup
% Remove Titles eg "Literatur"
\renewcommand{\section}[2]{}%
% Reminder: Recreate "SmartFactory.bll" when "citavi/citavi.bib" gets changed => run BibTeX ([F11] in TexMaker)
\bibliography{citavi/citavi}
% Use the "IEEE standard" as style => "IEEEtran.bst"
\bibliographystyle{IEEEtran}
\endgroup  
  

\end{document}
